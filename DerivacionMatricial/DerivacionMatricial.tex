%%%%%%%%%%%%%%%%%%%%%%%%%%%%%%%%%%%%%%%%%%%%%%%%%%%%%%%%%%%%%%%%%%%%%%%%%%%%%%%%%%%%%%%%%%%%%%%%%%%%%%
% Plantilla básica de Latex en Español.
%
% Autor: Andrés Herrera Poyatos (https://github.com/andreshp) 
%
% Es una plantilla básica para redactar documentos. Utiliza el paquete fancyhdr para darle un
% estilo moderno pero serio.
%
% La plantilla se encuentra adaptada al español.
%
%%%%%%%%%%%%%%%%%%%%%%%%%%%%%%%%%%%%%%%%%%%%%%%%%%%%%%%%%%%%%%%%%%%%%%%%%%%%%%%%%%%%%%%%%%%%%%%%%%%%%%

%-----------------------------------------------------------------------------------------------------
%	INCLUSIÓN DE PAQUETES BÁSICOS
%-----------------------------------------------------------------------------------------------------

\documentclass{article}

\usepackage{lipsum}                     % Texto dummy. Quitar en el documento final.

%-----------------------------------------------------------------------------------------------------
%	SELECCIÓN DEL LENGUAJE
%-----------------------------------------------------------------------------------------------------

% Paquetes para adaptar Látex al Español:
\usepackage[spanish,es-noquoting, es-tabla, es-lcroman]{babel} % Cambia 
\usepackage[utf8]{inputenc}                                    % Permite los acentos.
\selectlanguage{spanish}                                       % Selecciono como lenguaje el Español.

%-----------------------------------------------------------------------------------------------------
%	SELECCIÓN DE LA FUENTE
%-----------------------------------------------------------------------------------------------------

% Fuente utilizada.
\usepackage{courier}                    % Fuente Courier.
\usepackage{microtype}                  % Mejora la letra final de cara al lector.

%-----------------------------------------------------------------------------------------------------
%	ESTILO DE PÁGINA
%-----------------------------------------------------------------------------------------------------

% Paquetes para el diseño de página:
\usepackage{fancyhdr}               % Utilizado para hacer títulos propios.
\usepackage{lastpage}               % Referencia a la última página. Utilizado para el pie de página.
\usepackage{extramarks}             % Marcas extras. Utilizado en pie de página y cabecera.
\usepackage[parfill]{parskip}       % Crea una nueva línea entre párrafos.
\usepackage{geometry}               % Asigna la "geometría" de las páginas.

\usepackage{enumerate} 				% Para cambiar formato enumeración

% Se elige el estilo fancy y márgenes de 3 centímetros.
\pagestyle{fancy}
\geometry{left=3cm,right=3cm,top=3cm,bottom=3cm,headheight=1cm,headsep=0.5cm} % Márgenes y cabecera.
% Se limpia la cabecera y el pie de página para poder rehacerlos luego.
\fancyhf{}

% Espacios en el documento:
\linespread{1.1}                        % Espacio entre líneas.
\setlength\parindent{0pt}               % Selecciona la indentación para cada inicio de párrafo.

% Cabecera del documento. Se ajusta la línea de la cabecera.
\renewcommand\headrule{
	\begin{minipage}{1\textwidth}
		\hrule width \hsize 
	\end{minipage}
}

% Texto de la cabecera:
\lhead{}                          % Parte izquierda.
\chead{}                                    % Centro.
\rhead{\subject \ - \doctitle}              % Parte derecha.

% Pie de página del documento. Se ajusta la línea del pie de página.
\renewcommand\footrule{                                 
	\begin{minipage}{1\textwidth}
		\hrule width \hsize   
	\end{minipage}\par
}

\lfoot{}                                                 % Parte izquierda.
\cfoot{}                                                 % Centro.
\rfoot{Página\ \thepage\ de\ \protect\pageref{LastPage}} % Parte derecha.

%----------------------------------------------------------------------------------------
%	MATEMÁTICAS
%----------------------------------------------------------------------------------------

% Paquetes para matemáticas:                     
\usepackage{amsmath, amsthm, amssymb, amsfonts, amscd} % Teoremas, fuentes y símbolos.

% Nuevo estilo para definiciones
\newtheoremstyle{definition-style} % Nombre del estilo
{6pt}                % Espacio por encima
{6pt}                % Espacio por debajo
{}                   % Fuente del cuerpo
{}                   % Identación: vacío= sin identación, \parindent = identación del parráfo
{\bf}                % Fuente para la cabecera
{.}                  % Puntuación tras la cabecera
{.5em}               % Espacio tras la cabecera: { } = espacio usal entre palabras, \newline = nueva línea
{}                   % Especificación de la cabecera (si se deja vaía implica 'normal')

% Nuevo estilo para teoremas
\newtheoremstyle{theorem-style} % Nombre del estilo
{6pt}                % Espacio por encima
{0pt}                % Espacio por debajo
{\itshape}           % Fuente del cuerpo
{}                   % Identación: vacío= sin identación, \parindent = identación del parráfo
{\bf}                % Fuente para la cabecera
{.}                  % Puntuación tras la cabecera
{.5em}               % Espacio tras la cabecera: { } = espacio usal entre palabras, \newline = nueva línea
{}                   % Especificación de la cabecera (si se deja vaía implica 'normal')

% Nuevo estilo para ejemplos y ejercicios
\newtheoremstyle{example-style} % Nombre del estilo
{5pt}                % Espacio por encima
{0pt}                % Espacio por debajo
{}                   % Fuente del cuerpo
{}                   % Identación: vacío= sin identación, \parindent = identación del parráfo
{\scshape}                % Fuente para la cabecera
{:}                  % Puntuación tras la cabecera
{.5em}               % Espacio tras la cabecera: { } = espacio usal entre palabras, \newline = nueva línea
{}                   % Especificación de la cabecera (si se deja vaía implica 'normal')

% Teoremas:
\theoremstyle{theorem-style}  % Otras posibilidades: plain (por defecto), definition, remark
\newtheorem{theorem}{Teorema}[section]  % [section] indica que el contador se reinicia cada sección
\newtheorem{corollary}[theorem]{Corolario} % [theorem] indica que comparte el contador con theorem
\newtheorem{lemma}[theorem]{Lema}
\newtheorem{proposition}[theorem]{Proposición}

% Definiciones, notas, conjeturas
\theoremstyle{definition}
\newtheorem{definition}{Definición}[section]
\newtheorem{conjecture}{Conjetura}[section]
\newtheorem*{note}{Nota} % * indica que no tiene contador

% Ejemplos, ejercicios
\theoremstyle{example-style}
\newtheorem{example}{Ejemplo}[section]
\newtheorem{exercise}{Ejercicio}[section]

%-----------------------------------------------------------------------------------------------------
%	PORTADA
%-----------------------------------------------------------------------------------------------------

% Elija uno de los siguientes formatos.
% No olvide incluir los archivos .sty asociados en el directorio del documento.
\usepackage{title1}
%\usepackage{title2}
%\usepackage{title3}

%-----------------------------------------------------------------------------------------------------
%	TÍTULO, AUTOR Y OTROS DATOS DEL DOCUMENTO
%-----------------------------------------------------------------------------------------------------

% Título del documento.
\newcommand{\doctitle}{Derivación matricial}
% Subtítulo.
\newcommand{\docsubtitle}{Trabajo B}
% Fecha.
\newcommand{\docdate}{10 \ de \ enero \ de \ 2018}
% Asignatura.
\newcommand{\subject}{Estadística Multivariante}
% Autor.
\newcommand{\docauthor}{Antonio R. Moya Martín-Castaño \\Elena Romero Contreras \\Nuria Rodríguez Barroso}
\newcommand{\docaddress}{Universidad de Granada}
\newcommand{\docemail}{anmomar85@correo.ugr.es \\ elenaromeroc@correo.ugr.es \\ rbnuria6@gmail.com}

%-----------------------------------------------------------------------------------------------------
%	RESUMEN
%-------------------------------					----------------------------------------------------------------------

% Resumen del documento. Va en la portada.
% Puedes también dejarlo vacío, en cuyo caso no aparece en la portada.
%\newcommand{\docabstract}{}
\newcommand{\docabstract}{}

\begin{document}

 \maketitle

%-----------------------------------------------------------------------------------------------------
%	ÍNDICE
%-----------------------------------------------------------------------------------------------------

% Profundidad del Índice:
%\setcounter{tocdepth}{1}

\newpage
\tableofcontents
\newpage

\section{Introducción}

En este apartado se va a tratar la derivación respecto a vectores y matrices, que es muy necesaria en estadística multivariante sobre todo desde el punto de vista de la optimización. Así, permite calcular datos tales como estimador máximo verosímil, matrices de información de Fisher, o cotas tipo Crámer-Rao. Más importancia tiene todavía este tema si tenemos en cuenta que, si ya la derivación vectorial puede dar lugar a cálculos costosos, en el caso de la matricial se pueden generar un enorme número de derivadas que pueden resultar difícil de ordenar con sentido en una matriz. \\

Muchas son las aproximaciones que se han dado y en este caso lo que se hará es seguir una línea en la que las matrices se conviertan en vectores, mucho más fáciles de manejar. **********

\section{Diferencial primera y jacobianos}

\subsection{Diferencial de una función vectorial}
\textbf{Definición 2.2.1}: Consideramos una función vectorial $f: S \rightarrow \mathbb{R}^m$ con $S\subset \mathbb{R}^n$. Sea \textbf{c} un punto interior de S y consideremos una bola cerrada con centro en \textbf{c} y radio \textbf{r}, $B(c,r)$. Sea $u$ un punto de $\mathbb{R}^n$ tal que $||u||\leq r$ es decir, $c+r \in B(c,r)$. \\
Diremos que f es diferenciable en \textbf{c} si existe una matriz real de orden $m$ x $n$ que depende de $c$ y no de $u$ y que cumple que $f(c+u)-f(c) = A(c)u + r_c(u)$ con $\lim_{u\to0} \frac{r_c(u)}{||u||} = 0$. Además, se define la primera diferencial de $f$ en el punto $c$ con incremento $u$ como: $df(c;u)=A(c)u$.\\

\textbf{Definición 2.2.2}: Sea $f: S\subset \mathbb{R}^n \rightarrow \mathbb{R}^m$ y sea $f_i: S \rightarrow \mathbb{R}$ su i-ésima componente. Sea $c$ un punto interior de S y $e_j$ el j-ésimo vector de la base canónica de $\mathbb{R}^n$. Se define la derivada parcial de $f$ respecto a la j-ésima coordenada como: 

$$ D_jf_i(c) = \lim_{t\to0} \frac{f_i(c + te_j ) - f_i(c)}{t}, t\in \mathbb{R} $$  
 
\begin{theorem} (Primer teorema de identificación para funciones vectoriales)\\
	Sea $f: S\subset \mathbb{R}^n \rightarrow \mathbb{R}^m$ diferenciable en un punto $c$ interior de S y $u  \in \mathbb{R}^n$. Entonces $df(c;u)=(Df(c))u$ donde $Df(c)$ es una matriz $m$x$n$ cuyos elementos $D_jf_i(c)$ son las derivadas parciales de \textbf{f} evaluadas en $c$ y que recibe el nombre de matriz jacobiana. Recíprocamente, si $A(c)$ es una matriz que verifica que $df(c;u)=A(c)u \forall u\in \mathbb{R}^n$, entonces $A(c) = Df(c)$. 
\end{theorem}

\begin{theorem}
	Sea $f: S\subset \mathbb{R}^n \rightarrow \mathbb{R}^m$ diferenciable en un punto $c$ interior de S. Sea $T$ un subconjunto de $\mathbb{R}^m$ tal que $f(x) \in T \forall x \in S$ y supongamos que $g: T \rightarrow \mathbb{R}^p$ es diferenciable en un punto \textbf{b} (b = f(c)) de T. Entonces la función compuesta $h: S \rightarrow \mathbb{R}^p$ definida por $h(x)=g(f(x))$, es diferenciable en $c$ y $Dh(c)=Dg(b)Df(c)$.
\end{theorem}

\begin{theorem} (Regla de invarianza de Cauchy) \\
	En el ambiente del teorema anterior, si \textbf{f} es diferenciable en \textbf{c} y \textbf{g} lo es en \textbf{b=f(c)}, entonces la diferenciable $h=g$o$f$ es $dh(c;u) = dg(b;df(c;u))$.   
	
\end{theorem}

\begin{exercise}
	Sea $h: \mathbb{R}^k \rightarrow \mathbb{R}$ definida por $h(\beta) = (y-X\beta)^t(y-X\beta)$ donde $y \in \mathbb{R}^n$ y $X\in\mathbb{M}_{n\times k}.$ Haciendo uso de la regla de invarianza de Cauchy demostrar que
	$$dh(c;u) = dg(y-Xc;df(c;u)) = dg(y-Xc;-Xu)=-2(y-Xc)^tXu$$
	y con ello $Dh(c)=-2(y-Xc)^tX$.
\end{exercise}
\textit{Solución:}
	Para comenzar, tengamos en cuenta que h es composición de $f: \mathbb{R}^k \rightarrow \mathbb{R}^n$ y $g:\mathbb{R}^n \rightarrow \mathbb{R}$, con $f(c) = (y-Xc)$ con $X$ una matriz $nxk$ y con $y \in \mathbb{R}^n$; y $g(X) = X^tX$, de modo que $h(x)=g(f(x)) \forall x \in  \mathbb{R}^k$. Así, es claro que  $dh(c;u) = dg(y-Xc;df(c;u))$. ************************ Parece evidente pero no se que escribir xd****************
\begin{exercise}
	Sea $F(X)=AG(X)B$, donde $A_{m\times r}$ y $B_{s\times p}$ son matrices constantes y $G(X)_{r\times s}$ es una función diferenciable. Calcular $DF(C)$ a partir de la definición de diferencial matricial.
\end{exercise}
\textit{Solución:}

\begin{exercise}
	Si $X_{n\times n}$ es una matriz simétrica y $F: \mathbb{M}_{n\times q} \rightarrow \mathbb{M}_{m\times p}$ es diferenciable, demostrar que $d\text{Vec}(F(X)) = D_nDF(X)d\text{Vech}(X)$, mientras que $d\text{Vec}(F(X)) = N_nDF(X)d\text{Vec}(X)$ donde $N_n = \frac{1}{2}[I_{n^2}+K_{nn}]$.
\end{exercise}
\textit{Solución:}


\section{Matrices jacobianas y derivadas matriciales}

\begin{exercise}
	A partir de las relacioens existentes entre la derivada matricial y la matriz jacobiana, verificar las siguientes expresiones:
	\begin{enumerate}[a)]
		\item Sea $X_{n\times n}$ y $F(X) = \text{tr}[X].$ Entonces $DF(X) = \text{Vec}^t(I_n).$
		\item Sea ahora $X_{n\times q}$ y $F(X) =X$. Entonces $DF(X) = I_q \otimes I_n = I_{nq}$.
	\end{enumerate}
\end{exercise}

\begin{exercise}
	Sea $X_{n\times q}$. Demostra las siguientes igualdades:
	\begin{enumerate}[a)]
		\item $\displaystyle \frac{\partial X^t}{\partial X} = K_{qn}$.
		\item $\displaystyle \frac{\partial X}{\partial X^t} = K_{nq}$.
		\item $\displaystyle \frac{\partial X^t}{\partial X^t} = \text{Vec}(I_q)\text{Vec}^t(I_n)$.
	\end{enumerate}
\end{exercise}

\begin{exercise}
	Demostrar que si $X_{n\times n}$ es no singular entonces $\displaystyle \frac{\partial X^{-1}}{\partial X} = -\text{Vec}((X^{-1})^t)\text{Vec}^t(X^{-1})$.
\end{exercise}
\textit{Solución:}


\section{Diferencial segunda y hessianos}


	
\end{document}