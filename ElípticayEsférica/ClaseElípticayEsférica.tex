%%%%%%%%%%%%%%%%%%%%%%%%%%%%%%%%%%%%%%%%%%%%%%%%%%%%%%%%%%%%%%%%%%%%%%%%%%%%%%%%%%%%%%%%%%%%%%%%%%%%%%
% Plantilla básica de Latex en Español.
%
% Autor: Andrés Herrera Poyatos (https://github.com/andreshp) 
%
% Es una plantilla básica para redactar documentos. Utiliza el paquete fancyhdr para darle un
% estilo moderno pero serio.
%
% La plantilla se encuentra adaptada al español.
%
%%%%%%%%%%%%%%%%%%%%%%%%%%%%%%%%%%%%%%%%%%%%%%%%%%%%%%%%%%%%%%%%%%%%%%%%%%%%%%%%%%%%%%%%%%%%%%%%%%%%%%

%-----------------------------------------------------------------------------------------------------
%	INCLUSIÓN DE PAQUETES BÁSICOS
%-----------------------------------------------------------------------------------------------------

\documentclass{article}

\usepackage{lipsum}                     % Texto dummy. Quitar en el documento final.

%-----------------------------------------------------------------------------------------------------
%	SELECCIÓN DEL LENGUAJE
%-----------------------------------------------------------------------------------------------------

% Paquetes para adaptar Látex al Español:
\usepackage[spanish,es-noquoting, es-tabla, es-lcroman]{babel} % Cambia 
\usepackage[utf8]{inputenc}                                    % Permite los acentos.
\selectlanguage{spanish}                                       % Selecciono como lenguaje el Español.

%-----------------------------------------------------------------------------------------------------
%	SELECCIÓN DE LA FUENTE
%-----------------------------------------------------------------------------------------------------

% Fuente utilizada.
\usepackage{courier}                    % Fuente Courier.
\usepackage{microtype}                  % Mejora la letra final de cara al lector.

%-----------------------------------------------------------------------------------------------------
%	ESTILO DE PÁGINA
%-----------------------------------------------------------------------------------------------------

% Paquetes para el diseño de página:
\usepackage{fancyhdr}               % Utilizado para hacer títulos propios.
\usepackage{lastpage}               % Referencia a la última página. Utilizado para el pie de página.
\usepackage{extramarks}             % Marcas extras. Utilizado en pie de página y cabecera.
\usepackage[parfill]{parskip}       % Crea una nueva línea entre párrafos.
\usepackage{geometry}               % Asigna la "geometría" de las páginas.

% Se elige el estilo fancy y márgenes de 3 centímetros.
\pagestyle{fancy}
\geometry{left=3cm,right=3cm,top=3cm,bottom=3cm,headheight=1cm,headsep=0.5cm} % Márgenes y cabecera.
% Se limpia la cabecera y el pie de página para poder rehacerlos luego.
\fancyhf{}

% Espacios en el documento:
\linespread{1.1}                        % Espacio entre líneas.
\setlength\parindent{0pt}               % Selecciona la indentación para cada inicio de párrafo.

% Cabecera del documento. Se ajusta la línea de la cabecera.
\renewcommand\headrule{
	\begin{minipage}{1\textwidth}
		\hrule width \hsize 
	\end{minipage}
}

% Texto de la cabecera:
\lhead{}                          % Parte izquierda.
\chead{}                                    % Centro.
\rhead{\subject \ - \doctitle}              % Parte derecha.

% Pie de página del documento. Se ajusta la línea del pie de página.
\renewcommand\footrule{                                 
	\begin{minipage}{1\textwidth}
		\hrule width \hsize   
	\end{minipage}\par
}

\lfoot{}                                                 % Parte izquierda.
\cfoot{}                                                 % Centro.
\rfoot{Página\ \thepage\ de\ \protect\pageref{LastPage}} % Parte derecha.

%----------------------------------------------------------------------------------------
%	MATEMÁTICAS
%----------------------------------------------------------------------------------------

% Paquetes para matemáticas:                     
\usepackage{amsmath, amsthm, amssymb, amsfonts, amscd} % Teoremas, fuentes y símbolos.

% Nuevo estilo para definiciones
\newtheoremstyle{definition-style} % Nombre del estilo
{6pt}                % Espacio por encima
{6pt}                % Espacio por debajo
{}                   % Fuente del cuerpo
{}                   % Identación: vacío= sin identación, \parindent = identación del parráfo
{\bf}                % Fuente para la cabecera
{.}                  % Puntuación tras la cabecera
{.5em}               % Espacio tras la cabecera: { } = espacio usal entre palabras, \newline = nueva línea
{}                   % Especificación de la cabecera (si se deja vaía implica 'normal')

% Nuevo estilo para teoremas
\newtheoremstyle{theorem-style} % Nombre del estilo
{6pt}                % Espacio por encima
{0pt}                % Espacio por debajo
{\itshape}           % Fuente del cuerpo
{}                   % Identación: vacío= sin identación, \parindent = identación del parráfo
{\bf}                % Fuente para la cabecera
{.}                  % Puntuación tras la cabecera
{.5em}               % Espacio tras la cabecera: { } = espacio usal entre palabras, \newline = nueva línea
{}                   % Especificación de la cabecera (si se deja vaía implica 'normal')

% Nuevo estilo para ejemplos y ejercicios
\newtheoremstyle{example-style} % Nombre del estilo
{5pt}                % Espacio por encima
{0pt}                % Espacio por debajo
{}                   % Fuente del cuerpo
{}                   % Identación: vacío= sin identación, \parindent = identación del parráfo
{\scshape}                % Fuente para la cabecera
{:}                  % Puntuación tras la cabecera
{.5em}               % Espacio tras la cabecera: { } = espacio usal entre palabras, \newline = nueva línea
{}                   % Especificación de la cabecera (si se deja vaía implica 'normal')

% Teoremas:
\theoremstyle{theorem-style}  % Otras posibilidades: plain (por defecto), definition, remark
\newtheorem{theorem}{Teorema}[section]  % [section] indica que el contador se reinicia cada sección
\newtheorem{corollary}[theorem]{Corolario} % [theorem] indica que comparte el contador con theorem
\newtheorem{lemma}[theorem]{Lema}
\newtheorem{proposition}[theorem]{Proposición}

% Definiciones, notas, conjeturas
\theoremstyle{definition}
\newtheorem{definition}{Definición}[section]
\newtheorem{conjecture}{Conjetura}[section]
\newtheorem*{note}{Nota} % * indica que no tiene contador

% Ejemplos, ejercicios
\theoremstyle{example-style}
\newtheorem{example}{Ejemplo}[section]
\newtheorem{exercise}{Ejercicio}[section]

%-----------------------------------------------------------------------------------------------------
%	PORTADA
%-----------------------------------------------------------------------------------------------------

% Elija uno de los siguientes formatos.
% No olvide incluir los archivos .sty asociados en el directorio del documento.
\usepackage{title1}
%\usepackage{title2}
%\usepackage{title3}

%-----------------------------------------------------------------------------------------------------
%	TÍTULO, AUTOR Y OTROS DATOS DEL DOCUMENTO
%-----------------------------------------------------------------------------------------------------

% Título del documento.
\newcommand{\doctitle}{Clase esférica y elíptica  de distribuciones}
% Subtítulo.
\newcommand{\docsubtitle}{Trabajo A}
% Fecha.
\newcommand{\docdate}{20 \ de \ Diciembre \ de \ 2017}
% Asignatura.
\newcommand{\subject}{Estadística Multivariante}
% Autor.
\newcommand{\docauthor}{Antonio R. Moya Martín-Castaño \\Elena Romero Contreras \\Nuria Rodríguez Barroso}
\newcommand{\docaddress}{Universidad de Granada}
\newcommand{\docemail}{anmomar85@correo.ugr.es \\ elenaromeroc@correo.ugr.es \\ rbnuria6@gmail.com}

%-----------------------------------------------------------------------------------------------------
%	RESUMEN
%-------------------------------					----------------------------------------------------------------------

% Resumen del documento. Va en la portada.
% Puedes también dejarlo vacío, en cuyo caso no aparece en la portada.
%\newcommand{\docabstract}{}
\newcommand{\docabstract}{}

\begin{document}

 \maketitle

%-----------------------------------------------------------------------------------------------------
%	ÍNDICE
%-----------------------------------------------------------------------------------------------------

% Profundidad del Índice:
%\setcounter{tocdepth}{1}

\newpage
\tableofcontents
\newpage
	
%----------------------------------------------------------------------------------------
%	Sección 1: Deficiones y teoremas
%----------------------------------------------------------------------------------------

\section{Introducción.}

	La distribución normal multivariante que hemos desarrollado en clase de teoría, es un caso particular de una familia de distribuciones muy utilizadas en el análsis multivariante, las \textit{distribuciones elípticas}. Para introducirlas, consideraremos en primer lugar el caso más simple de estas, las \textit{distribuciones esféricas}.
	
	Para finalizar, veremos casos concretos de distribuciones de estas clases en $\mathbb{R}^p$.

\section{Clases esférica y elíptica de distribuciones en $\mathbb{R}^p$}
	
	\begin{definition}
		Dado un vector aleatorio \textbf{X} = $(X_1, ... , X_p)^t$, se dice que se distribuye en la clase esférica de distribuciones en $\mathbb{R}^p$ si \textbf{X} y \textbf{HX} tienen la misma distribución, $\forall$\textbf{H} $\in O(p)$ siendo $O(p)$ el grupo de matrices ortogonales de orden p. Esto es, si su distribución es invariante frente a transformaciones ortogonales. 
	\end{definition}

	

	\begin{example}
		Un ejemplo de densidad esférica es la distribución uniforme en la hiperesfera de radio r:
		
		\begin{center}
		\begin{math}
				f_x(\textbf{x}) = \frac{\Gamma(\frac{p+2}{2})}{\pi ^ {p/2} r^p} \textbf{I}_{[\textbf{x}^t\textbf{x} \leq r^2]}
		\end{math}
		\end{center}
	\end{example}


	Vemos ahora un resultado sobre la forma que adopta la función característica de cualquier variable que se distribuya en la clase esférica.
	
	\begin{theorem}
		Sea \textbf{X} un vector aleatorio con distribución en la clase esférica. Entonces su función característica es de la forma $\phi($\textbf{t}$) = \psi$(\textbf{t}$^t$\textbf{t}), con $\psi$ una cierta función. Además, E\textbf{[X] = 0} y Cov[\textbf{X}]  = $-2\psi'(0)$\textbf{I}$_p$.
	\end{theorem}

	\begin{proof}
		
		Para la primera parte de esta demostración vamos a utilizar resultados de la teoría de invarianzas:
		
		\begin{definition}
			Se dice que una función $\Phi$ es invariante sobre el espacio $\chi$ es invariante bajo G si verifica $\Phi(gx) = \Phi(x), \forall x \in \chi, \forall g \in G$
		\end{definition}
	
		\begin{definition}
			Una función $\Phi$ sobre $\chi$ se dice invariante maximal bajo G si es invariante bajo G y además verifica $\phi(x_1) = \phi(x_2) \Rightarrow x_1 \sim x_2 (mod G)$
		\end{definition}
	
		\begin{theorem}
			Sea $\Phi$ un invariante maximal bajo G para $\chi$. Entonces una función $\psi$ sobre $\chi$ es invariante bajo G sí y solo sí es función de $\Phi$.
		\end{theorem}
	
	\begin{proposition}
		Sea $G = O(p)$ el grupo  de matrices ortogonales de orden $p \times p$. Entonces $\Phi(x) = x^tx$ es un invariante maximal bajo G.
	\end{proposition}

	Para ver que $\Phi(t) = \psi(t^tt)$, basta con considerar la función $f(t) = t^tt$, que es un invariante maximal por la proposición anterior, y que $\Phi$ es un invariante por la definición de clase esférica (como $X$ y $HX$ tienen la misma distribución, $\Phi_x(t) = \Phi_x(Ht), \forall H \in G$). Entonces, aplicando el Teorema anterior, tenemos que $\Phi(t) = \psi(f(t)) = \psi(t^tt)$, para alguna función $\psi$.
	
	Para demostrar que $E[X] = 0$, basta con darse cuenta de que, dado que $X$ y $HX$  tienen la misma distribución, se verifica que $E[X] = E[HX], \forall H \in O(p)$,  y por la linealidad de la esperanza matemática sabemos que $E[HX] = H E[X], \forall H \in O(p)$, por tanto, llegamos a que $E[X] = HE[X], \forall H \in O(p)$, luego se debe verificar que $E[X] = 0$.
	
	********************** FALTA LO DE LA COVARIANZA
	
	\end{proof}

	\begin{definition}
		Dado un vector aleatorio \textbf{X} = $(X_1, ..., X_p)^t$, se dice que se distribuye en la clase elíptica de parámetros $\mu \in \mathbb{R}^p$ y \textbf{V}$_{p \times p}$ (\textbf{V} $> 0$) si su densidad es de la forma
		
		\begin{center}
		\begin{math}
			f(\textbf{x}) = C_p|\textbf{V}|^{-\frac{1}{2}}h((\textbf{x}-\mu)^t\textbf{V}^{-1}(\textbf{x}-\mu))
		\end{math}
		\end{center}
		donde $C_p$ es una constante y h una función suficientemente regular. A la clase elíptica la notaremos por $E_p(\mu; V)$.
	\end{definition}

	\begin{example}
		Un ejemplo de densidad elíptica es la distribución uniforme en el elipsoide de dimensión p:
		
		\begin{center}
			\begin{math}
				f_x(\textbf{x}) = \frac{\Gamma(\frac{p+2}{2}|\textbf{A}|^\frac{1}{2})}{\pi^\frac{p}{2}r^0}\textbf{I}_{[(\textbf{x}-\mu)^t\textbf{A}(\textbf{x}-\mu)\leq r^2]}
			\end{math}
		\end{center}

	\end{example}

	\textbf{EJERCICIOS:}
		\begin{itemize}
			\item Comprobar que la clase esférica coincide con la elíptica $E(0; I_p )$.
			
			\textit{Solución:} 
			
			Si tenemos que un vector aleatorio \textbf{X} = $(X_1, ..., X_p)^t$ se distribuye en la clase elíptica $E(0; I_p )$, entonces cumple que su función de densidad es de la forma $f_x(\textbf{x}) = C_p h (x^tx)$, por lo que sólo depende de x a través de $x^tx$. De este modo, es invariantes por transformaciones ortogonales con lo que pertenece también a la clase esférica. ¿¿¿¿El OTRO CASO???
			
			\item Verificar la siguiente caracterización, que generaliza la vista en el caso normal: $$ X \in E_p(\mu; V) \leftrightarrows X = \mu + CU $$ donde $\mu \in \mathbb{R}^p$ , C es la matriz dada por la descomposición de Cholesky de V, es decir, $V=CC^t$ y $U$ es un vector de la clase esférica p-dimensional.
			
			\textit{Solución:}
			
				Consideramos $X = \mu + CU $, y calculamos cual es su función de distribución. Para ello, como $U$ es un vector de la clase esférica p-dimensional, consideramos $f_U$ su función de distribución. Por lo que ya hemos demostrado, $f_U$ depende de $t$ solo a través de $t^tt$, esto es $_u(tf)=h(t^tt)$,para cierta función $h$. Además, estamos considerando que la matriz C es no singular. Por tanto, la transformación de U a X tiene el Jacobiano $J = det(C^{-1})$, y tenemos que aplicando un resultado de transformación para vectores aleatorios podemos obtener la función de densidad de $X$ como:
				
				\[
					f_x(x) = |det(C^{-1})| f_U |C^{-1}(x-\mu)| = [det(A)^{-1}det(A^{-1})]^{1/2}h[(x-\mu)^t(A^t)^{-1}A^{-1}(x-\mu)] =
				\]
								
				\[
									det[(AA^t)^{-1}]^{1/2}h[(x-\mu)^t(AA^t)^{-1}(x-\mu)] = \frac{h[(x-\mu)^tV^{-1}(x-\mu)]}{|V|^{1/2}}				
				\]



			*********** ME LO HE SACADO DEL SEGUNDO LIBRO, MAGIA!
				

				
				
			\item Verificar que si $X \in E_p(\mu; V)$  entonces $E[X] = \mu + CE[U]$ y $Cov(X)= C Cov(U)C^t$ siendo U pertenciente a la clase esférica.
			
			\textit{Solución:}
			
			Dada la linealidad de la esperanza matemática, sabemos que $E[\mu + CU] = E[\mu] + E[CU] = \mu + CE[U]$.
			
			En cuanto a la covarianza, dada la invarianza por translaciones obtenemos que $Cov[\mu + CU] = Cov[CU]$, y aplicando la bilinealidad de la covarianza obtenemos $Cov[CU] = C Cov[U]C^t$, por lo que hemos obtenido que $Cov[\mu + CU] = C Cov[U] C^t$.

		\end{itemize}
		
		En cuanto a la función característica, se plantea el siguiente ejercicio:
		
		\begin{itemize}
			\item Sea \textbf{X} un vector aleatorio con distribución en la clase elíptica $E_p(\mu; V)$. Entonces su función característica es de la forma $\phi(u)=e^{iu^t\mu}\psi(u^tVu)$, con $\psi$ una cierta función.		
			
			\textit{Solución:}
			
			Sea $X \in E_p(\mu,V)$, aplicando al definición de función característica tenemos
			\[
				\phi_X(u) = \phi(u) = E[exp(iu^tX)] = E[exp(iu^t(\mu + CU))] = exp(i u^t \mu) \phi_{CU}(u) = exp(iu^t\mu)\phi_U(C^t u)
			\]
			
			Aplicando ahora el primer resultado que hemos demostrado ($\phi_u(t) = \psi(t^tt)$) para una cierta función $\chi$ tenemos
			
			\[
				\phi(u) = exp(iu^t\mu)\psi(t^tCC^tt) = exp(i u^t \mu) \psi(t^t V t)
			\]
			 
			 como buscábamos.
			
			\item Sea \textbf{X} un vector aleatorio con distribución en la clase elíptica $E_p(\mu; V)$. Comprobar que si $A_{qxp}$ es una matriz de constantes con rg(A)=$q\leq p$, y c$\in\mathbb{R}^p$, entonces \textbf{Y = c + AX}$\in E_q(c+A\mu; AVA^t)$.
			
			\textit{Solución: }
			
			
			
			\item Sea \textbf{X}$\in E_p(\mu; V)$. Entonces $E[X]=\mu$ y $Cov[X]=-2\psi'(0)$V.
			
			\textit{Solución: }
			
			Para resolver este ejercicio, basta con utilizar el ejercicio anterior en el que se expresaba la esperanza matemática y la matriz de  correlaciones de un vector aleatorio con distribución elíptica $X = \mu + CU$ en función de la esperanza matemática y la matriz de correlaciones de $U$, siendo $U$ perteneciente a la clase esférica junto $E[u]$ y $Cov[U]$ calculadas anteriormente, de esta forma tenemos:
			\[
				E[X] = \mu + C E[U] = \mu + C 0 = \mu
			\]
			\[
				Cov[X] = C Cov[U] C^t = C (-2 \psi'(0)) I_p XC^t = -2 \psi'(0) C I_p C^t = -2 \psi'(0) V
			\]
			
			\item Comprobar que todas las distribuciones de la clase elíptica $E_p(\mu; V)$ tienen igual matriz de correlaciones.
			
			\textit{Solución:}
			
			Todas las distribuciones de la clase $X \in E_p(\mu;V)$ se pueden denotar de la siguiente manera $X = \mu + CU$, donde $U$ pertenece a la clase elítpica y $C$ verifica $V = CC^t$.
			
			Por tanto, hemos probado en el apartado anterior que $Cov[X] = -2 \psi'(0) V$.
			
			**** PUTADA, la $\psi$ no es la misma para todas las esféricas, o si? Si no ver que $\psi'(0)$ si que lo es.
			
		\end{itemize}
		
		\textbf{EJERCICIO:}
		
		Sea \textbf{X}$\in E_p(\mu; V)$. Particionemos el vector \textbf{X} de la forma \textbf{X}=$(X^t_{(1)}|X^t_{(2)})^t$ donde $X_{(1)}$ es de dimensión $q$x1 y $X_{(2)}$ lo es $(p-q)$x1. Consideremos en $\mu$ y \textbf{V} las particiones inducidas
		$$\mu = \left( \begin{array}{c}
						\mu_{(1)} \\ \mu_{(2)}
						\end{array}\right);
		V = \left( \begin{array}{cc}
					V_{11} & V_{12} \\ V_{21} & V_{22}
					\end{array}\right)$$
		Entonces se verifica
		\begin{itemize}
			\item $X_{(1)} \in E_q(\mu_{(1)}; V_{11})$
			\item $X_{(2)} \in E_{(p-q)}(\mu_{(2)}; V_{22})$
		\end{itemize}
		
		\textbf{Nota}: Hacerlo usando la función característica y también mediante la caracterización obtenida en el primer ejercicio.
		
		\begin{proof}
			
			\textit{Demostración usando la función característica: } 
			
			Basta con aplicar la definición de función característica anteriormente considerada tomando $u = (u_1^t: 0^t)^t$, donde $u_1$ es de dimensión qx1. Así, tenemos que
			\[
				\phi_{X_1}(u_1) = exp(i u_1^t \mu_1^t) \psi(u_1^t V_{11} u_1)
			\]
			
			que es la función característica de un vector aleatorio con distribución elíptica $E_q(\mu_1, V_{11})$.
			
			De forma análoga, tomando $u = (0^t; u_2^t)$, donde $u_2$ es de dimensión (p-1)x1, otenemos
			\[
							\phi_{X_2}(u_2) = exp(i u_2^t \mu_2^t) \psi(u_2^t V_{2} u_2)
			\]
			
			
			\textit{Con la caracterización obtenida en el primer ejercicio}
			$rg(A) = q \leq p$ y $c \in R^p$.
			

		Vamos a demostrarlo hora utilizando que $Y = c + AX \in E_q(c + A \mu; A V A^t)$ , donde  $rg(A) = q \leq p$ y $c \in \mathbb{R}^p$.
		
		Para este caso, tenemos que $X_1 = A_1 X$, donde $A_1 = \left( \begin{array}{cc}
		I_q & 0 \\ 0 & 0
		\end{array}\right)$

	
	  donde $rg(A_1) = rg(I_q) = q$ y $c = 0$, por tanto, aplicando la caracterización anterior, tenemos que $X_1 \in E_q(A_1 \mu; A V A^t) = E_q(\mu_1, V_{11})$.
	  
	  ********************* NO VEO QUE $V_{11} = A V A^t$, pero es que tiene que ser así.
	  
	  Análogamente, tenemos $X_2 = A_2 X$, donde $A_2 = \left( \begin{array}{cc}
	  0 & 0 \\ 0 & I_{(p-q)}
	  \end{array}\right)$
	  
	  Y en este caso, se verifica $rg(A_2) = p-q$ y $c = 0$, luego tenemos $X_2 \in E_{(p-q)}(A_2 \mu, A V A^t) = E_{(p-q)}(\mu_2, V_{22})$
	
	\end{proof}
	\section{Algunas distribuciones de las clases esférica y elíptica de distribuciones en $\mathbb{R}^p$ }
	
	\subsection{Distribución uniforme en el círculo de radio r}
	
	Sea $S=\{(x_1, x_2) \in \mathbb{R}^2: x_1^2+x_2^2 \leq r^2\}$ el círculo centrado en el origen y de radio $r>0$. Dado un vector aleatorio: \textbf{X} = $(X_1, X_2)^t$ se dice que sigue la distribución uniforme en $S^1$ si su densidad es $$f(x_1,x_2) = KI_{S^1}$$ donde $K>0$ e $I_{S^1}$ es la función indicadora en $S^1$.
	
	\textbf{EJERCICIOS:}
	
	\begin{itemize}
		\item Verificar que $K=\frac{1}{\pi r^2}$
		
		\textit{Solución:} Tengamos en cuenta que $\int_{-\infty}^{\infty} f_X(X) dX = 1$. Por tanto $K \int_{-\infty}^{\infty} I_{S^1} dX =1 $. Así: $$K \int_{S^1}^{}1dX = 1 $$ Y por tanto, atendiendo a que esa integral coincide con el área de un círculo: $K\pi r^2 = 1$. Y como conclusión sacamos que $K=\frac{1}{\pi r^2}$
		
		\item Comprobar que $E[X]=0$ y que $Cov[X] = \frac{r^2}{4}I_2$
		
		\textit{Solución:}
		$$E[X]=\int_{-\infty}^{+\infty}xf(x) dx = \int_{-\infty}^{+\infty}xKI_{S^1} dx = \int_{S^1}Kx dx = k \cdot \int_{S^1}x dx$$ 
		
		Realizamos cambio a polares:
		 $$x_1 = \rho cos(\theta), x_2 = \rho sen(\theta)$$
		donde $\rho>0$ y $0<\theta\leq2\pi$
		
		Definimos $\phi:]0,r[\times]0,2\pi[\longrightarrow S^1$ difeomorfismo con $\phi(\rho,\theta) = (\rho cos(\theta), \rho sen(\theta))$
		
		Calculamos el jacobiano:
		
		$$|Jac \phi|(\rho,\theta) = \left| \begin{array}{cc}
		cos(\theta) & -\rho sen(\theta) \\
		sen(\theta) & \rho cos(\theta)  \end{array} \right| = |\rho(cos^2(\theta)+sen^2(\theta))| = |\rho| = \rho$$
		
		$$\int_{S^1}f(x_1,x_2) dx = \int_{0}^{r}\int_{0}^{2\pi}x(f\circ \rho)(\rho,\theta)|Jac \phi|(\rho,\theta)d\theta d\rho = \int_{0}^{r}\int_{0}^{2\pi}(\rho^2cos(\theta),\rho^2sen(\theta))d\theta d\rho$$
		$$ \int_{0}^{r}\rho^2\left[(sen(\theta),-cos(\theta)\right]_0^{2\pi} d\rho = \int_0^{r^2}\rho^2\left[(0,1)-(0,-1)\right]d\rho = \int_0^{r^2}0d\rho = 0$$		
		
		Luego $E[X]=0$
		
		Veamos ahora que $Cov[X] = \frac{r^2}{4}I_2$. En primer lugar, comprobemos que $Var[X]=\left(\frac{r^2}{4},\frac{r^2}{4}\right)$
		
		$$\int_{-\infty}^{+\infty}x^2f(x_1,x_2)dx = \int_{-\infty}^{+\infty}x^2\cdot kI_{S^1}dx = k\int_{S^1}x^2dx$$
		Realizamos el cambio a polares anterior:
		$$k\cdot \int_{S^1}x^2dx = k \int_{0}^{r}\left(\int_{0}^{2\pi}(\rho^3cos^2(\theta),\rho^3sen^2(\theta))d\theta)d\rho\right) = k \int_{0}^{r}\rho^3\left(\int_0^{2\pi}(cos^2(\theta),sen^2(\theta)d\theta\right)d\rho$$
		$$=k \int_{0}^{r}\rho^3\left[\frac{1}{2}\left(\theta+sen(\theta)cos(\theta),\theta-sen(\theta)cos(\theta)\right)\right]^{2\pi}_{0} =k\int_{0}^{r}\rho^3(\pi,\pi)d\rho = (\pi,\pi)\int_{0}^{r}\rho^3 d\rho$$
		$$=k(\pi,\pi)\left[\frac{\rho^4}{4}\right]^r_0 = $$
		\textcolor{red}{*******}
		
		
		\item Calcular las distribuciones marginales así como las condicionadas. Por simetría basta calcular la distribución de $X_1$  y la de $X_1|X_2=x2$. 
		
		 
	\end{itemize} 
	
	
	\subsection{Distribución uniforme en la esfera de radio r}
	
	Sea $S^2=\{(x_1, x_2, x_3) \in  \mathbb{R}^3: x_1^2 + x_2^2+x_3^2\leq r^2\}$ el interior y borde de la esfera centrada en 0 y de radio $r>0$. Dado un vector aleatorio \textbf{X} = $(X_1, X_2, X_3)^t$ se dice que sigue la distribución uniforme en $S^2$ si su densidad es:
	$$ f(x_1, x_2, x_3)= KI_{S^2}$$
	
	donde $K>0$ e $I_{S^2}$ en la función indicadora en $S^2$. 
	
	\textbf{EJERCICIOS:}
	\begin{itemize}
		\item Verificar que $K=\frac{3}{4\pi r^3}$
		
		\textit{Solución:} De manera similar al caso del círculo, ahora tenemos que se cumple que $\int_{-\infty}^{\infty} f_X(X) dX = 1$. Por tanto $K \int_{-\infty}^{\infty} I_{S^2} dX =1 $. Así: $$K \int_{S^2}^{}1dX = 1 $$ Y por tanto, atendiendo a que esa integral coincide con el área de una esfera: $K\frac{4}{3}\pi r^3 = 1$. Y como conclusión sacamos que $K=\frac{3}{4\pi r^3}$
		
		\item   Comprobar que $E[X]=0$ y que $Cov[X] = \frac{r^2}{5}I_3$ ???????? 3???*********
		
		\item Calcular las distribuciones marginales unidimensionales y bidimensionales. Dada la simetría, basta con calcular la distribución de $X_1$ y de $(X_1, X_2)^t$
		
		\item Calcular las distribuciones de $(X_2, X_3)^t|X_1=x_1$ y $X_3|X_1=x_1, X_2=x_2$. Deducir que son distribuciones uniformes en un círculo y en un intervalo, respectivamente y, en consecuencia, calcular los dos primeros momentos.
		
	\end{itemize}
	
	\subsection{Distribución uniforme en la hiperesfera de radio r}
	
	Sea $S^{p-1} = \{x \in \mathbb{R}^p: x^tx\leq r^2\}$ el interior y el borde de la hiperesfera centrada en el origen y de radio $r>0$. Dado un vector aleatorio \textbf{X} = $(X_1, \dots , X_p)^t$ se dice que sigue la distribución uniforme en $S^{p-1}$ si su densidad es:
	
	$$f(x_1, \dots, x_p)= KI_{S^{p-1}}$$
	
	donde $K>0$ e $I_{S^{p-1}}$ es la función indicadora en $ S^{p-1}$
	
	\textbf{EJERCICIOS:}
	\begin{itemize}
		\item Verificar que $K = \frac{\Gamma(\frac{p+2}{2})}{\pi ^ {p/2} r^p}$.
		
		\item Comprobar que $E[X]=0$ y $Cov[X]=\frac{r^2}{p+2}I_p$.
		
		\item Calcular las distribuciones marginales unidimensionales y bidimensionales. Dada la simetría basta con calcular, por ejemplo, las de $X_1$ y $(X_1,X_2)^t$.
		
		\item Si consideramos \textbf{X} partido en la forma \textbf{X}=$(X^t_{(1)}|X^t_{(2)})^t$ donde $X_{(1)}$ es de dimensión $q$ y $X_{(2)}$ lo es $(p-q)$, calcular la distribución de $X_{(1)}$.
		
		\item Calcular la distribución condicionada $X_{(2)}|X_{(1)} = x_{(1)}$. Deducir que es una distribución uniforme en $S^{p-q-1}$, o sea, la esfera de dimensión $p-q$. En consecuencia, calcular los dos primeros momentos.
	\end{itemize}
	
	\subsection{Distribución T-Student esférica}
	
	Sea \textbf{X} = $(X_1, \dots , X_p)^t$ un vector aleatorio. Diremos que se distribuye según la distribución t de student esférica p-dimensional con n grados de libertad si su función de densidad es:
	$$ f(x_1, \dots, x_p)^t = \frac{\Gamma(\frac{n+p}{2})}{\Gamma(\frac{n}{2})(n \pi )^{\frac{p}{2}}[1+\frac{1}{n}x^tx]^{\frac{n+p}{2}}} $$ con $x \in \mathbb{R}^p$
	
	\textbf{EJERCICIO:}Demostrar que los momentos de esta distribución son:
	
	$$ E[X]=0$$ $$ Cov[X]= \frac{n}{n-2}I_{p}$$ con $n>2$. 
	
	\subsection{Versiones elípticas de las densidades uniformes en la hiperesfera y t-student}
	
	\textbf{EJERCICIOS:}
	\begin{itemize}
		\item Sea \textbf{U} un vector p-dimensional distribuido de forma uniforme en el interior y borde de la hiperesfera de dimensión \textit{p} y racio $r>0.$ Consideremos $\mu$ un vector de $\mathbb{R}^p$ y $V_{pxp}$ una matriz definida positiva descompuesta en la forma $V=CC^t$. Calcular la densidad del vector aleatorio $X=\mu+CU$.
		
		\item Calcular $E[X]$ y $Cov[X]$ para la distribución uniforme en el elipsoide $E^p_r$.
		
		\item Sea \textbf{U} un vector p-dimensional distribuido según una t de Studen multivariante esférica. Consideremos $\mu$ un vector de $\mathbb{R}^p$ y $V_{pxp}$ una matriz definida positiva descompuesta en la forma $V=CC^t$. Calcular la densidad del vector aleatorio $X=\mu+CU$. 
		
		\item Calcular $E[X]$ y $Cov[X]$ para la distribución anterior.
	\end{itemize}
	
\end{document}