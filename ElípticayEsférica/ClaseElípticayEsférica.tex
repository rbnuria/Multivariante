%%%%%%%%%%%%%%%%%%%%%%%%%%%%%%%%%%%%%%%%%%%%%%%%%%%%%%%%%%%%%%%%%%%%%%%%%%%%%%%%%%%%%%%%%%%%%%%%%%%%%%
% Plantilla básica de Latex en Español.
%
% Autor: Andrés Herrera Poyatos (https://github.com/andreshp) 
%
% Es una plantilla básica para redactar documentos. Utiliza el paquete fancyhdr para darle un
% estilo moderno pero serio.
%
% La plantilla se encuentra adaptada al español.
%
%%%%%%%%%%%%%%%%%%%%%%%%%%%%%%%%%%%%%%%%%%%%%%%%%%%%%%%%%%%%%%%%%%%%%%%%%%%%%%%%%%%%%%%%%%%%%%%%%%%%%%

%-----------------------------------------------------------------------------------------------------
%	INCLUSIÓN DE PAQUETES BÁSICOS
%-----------------------------------------------------------------------------------------------------

\documentclass{article}

\usepackage{lipsum}                     % Texto dummy. Quitar en el documento final.

%-----------------------------------------------------------------------------------------------------
%	SELECCIÓN DEL LENGUAJE
%-----------------------------------------------------------------------------------------------------

% Paquetes para adaptar Látex al Español:
\usepackage[spanish,es-noquoting, es-tabla, es-lcroman]{babel} % Cambia 
\usepackage[utf8]{inputenc}                                    % Permite los acentos.
\selectlanguage{spanish}                                       % Selecciono como lenguaje el Español.

%-----------------------------------------------------------------------------------------------------
%	SELECCIÓN DE LA FUENTE
%-----------------------------------------------------------------------------------------------------

% Fuente utilizada.
\usepackage{courier}                    % Fuente Courier.
\usepackage{microtype}                  % Mejora la letra final de cara al lector.

%-----------------------------------------------------------------------------------------------------
%	ESTILO DE PÁGINA
%-----------------------------------------------------------------------------------------------------

% Paquetes para el diseño de página:
\usepackage{fancyhdr}               % Utilizado para hacer títulos propios.
\usepackage{lastpage}               % Referencia a la última página. Utilizado para el pie de página.
\usepackage{extramarks}             % Marcas extras. Utilizado en pie de página y cabecera.
\usepackage[parfill]{parskip}       % Crea una nueva línea entre párrafos.
\usepackage{geometry}               % Asigna la "geometría" de las páginas.

% Se elige el estilo fancy y márgenes de 3 centímetros.
\pagestyle{fancy}
\geometry{left=3cm,right=3cm,top=3cm,bottom=3cm,headheight=1cm,headsep=0.5cm} % Márgenes y cabecera.
% Se limpia la cabecera y el pie de página para poder rehacerlos luego.
\fancyhf{}

% Espacios en el documento:
\linespread{1.1}                        % Espacio entre líneas.
\setlength\parindent{0pt}               % Selecciona la indentación para cada inicio de párrafo.

% Cabecera del documento. Se ajusta la línea de la cabecera.
\renewcommand\headrule{
	\begin{minipage}{1\textwidth}
		\hrule width \hsize 
	\end{minipage}
}

% Texto de la cabecera:
\lhead{\docauthor}                          % Parte izquierda.
\chead{}                                    % Centro.
\rhead{\subject \ - \doctitle}              % Parte derecha.

% Pie de página del documento. Se ajusta la línea del pie de página.
\renewcommand\footrule{                                 
	\begin{minipage}{1\textwidth}
		\hrule width \hsize   
	\end{minipage}\par
}

\lfoot{}                                                 % Parte izquierda.
\cfoot{}                                                 % Centro.
\rfoot{Página\ \thepage\ de\ \protect\pageref{LastPage}} % Parte derecha.

%----------------------------------------------------------------------------------------
%	MATEMÁTICAS
%----------------------------------------------------------------------------------------

% Paquetes para matemáticas:                     
\usepackage{amsmath, amsthm, amssymb, amsfonts, amscd} % Teoremas, fuentes y símbolos.

% Nuevo estilo para definiciones
\newtheoremstyle{definition-style} % Nombre del estilo
{5pt}                % Espacio por encima
{0pt}                % Espacio por debajo
{}                   % Fuente del cuerpo
{}                   % Identación: vacío= sin identación, \parindent = identación del parráfo
{\bf}                % Fuente para la cabecera
{.}                  % Puntuación tras la cabecera
{.5em}               % Espacio tras la cabecera: { } = espacio usal entre palabras, \newline = nueva línea
{}                   % Especificación de la cabecera (si se deja vaía implica 'normal')

% Nuevo estilo para teoremas
\newtheoremstyle{theorem-style} % Nombre del estilo
{5pt}                % Espacio por encima
{0pt}                % Espacio por debajo
{\itshape}           % Fuente del cuerpo
{}                   % Identación: vacío= sin identación, \parindent = identación del parráfo
{\bf}                % Fuente para la cabecera
{.}                  % Puntuación tras la cabecera
{.5em}               % Espacio tras la cabecera: { } = espacio usal entre palabras, \newline = nueva línea
{}                   % Especificación de la cabecera (si se deja vaía implica 'normal')

% Nuevo estilo para ejemplos y ejercicios
\newtheoremstyle{example-style} % Nombre del estilo
{5pt}                % Espacio por encima
{0pt}                % Espacio por debajo
{}                   % Fuente del cuerpo
{}                   % Identación: vacío= sin identación, \parindent = identación del parráfo
{\scshape}                % Fuente para la cabecera
{:}                  % Puntuación tras la cabecera
{.5em}               % Espacio tras la cabecera: { } = espacio usal entre palabras, \newline = nueva línea
{}                   % Especificación de la cabecera (si se deja vaía implica 'normal')

% Teoremas:
\theoremstyle{theorem-style}  % Otras posibilidades: plain (por defecto), definition, remark
\newtheorem{theorem}{Teorema}[section]  % [section] indica que el contador se reinicia cada sección
\newtheorem{corollary}[theorem]{Corolario} % [theorem] indica que comparte el contador con theorem
\newtheorem{lemma}[theorem]{Lema}
\newtheorem{proposition}[theorem]{Proposición}

% Definiciones, notas, conjeturas
\theoremstyle{definition}
\newtheorem{definition}{Definición}[section]
\newtheorem{conjecture}{Conjetura}[section]
\newtheorem*{note}{Nota} % * indica que no tiene contador

% Ejemplos, ejercicios
\theoremstyle{example-style}
\newtheorem{example}{Ejemplo}[section]
\newtheorem{exercise}{Ejercicio}[section]

%-----------------------------------------------------------------------------------------------------
%	PORTADA
%-----------------------------------------------------------------------------------------------------

% Elija uno de los siguientes formatos.
% No olvide incluir los archivos .sty asociados en el directorio del documento.
\usepackage{title1}
%\usepackage{title2}
%\usepackage{title3}

%-----------------------------------------------------------------------------------------------------
%	TÍTULO, AUTOR Y OTROS DATOS DEL DOCUMENTO
%-----------------------------------------------------------------------------------------------------

% Título del documento.
\newcommand{\doctitle}{Clase esférica y elíptica  de distribuciones}
% Subtítulo.
\newcommand{\docsubtitle}{Trabajo A}
% Fecha.
\newcommand{\docdate}{20 \ de \ Diciembre \ de \ 2017}
% Asignatura.
\newcommand{\subject}{Estadística Multivariante}
% Autor.
\newcommand{\docauthor}{Antonio R. Moya Martín-Castaño \\ Elena Romero Contreras \\ Nuria Rodríguez Barroso}
\newcommand{\docaddress}{Universidad de Granada}
\newcommand{\docemail}{anmomar85@correo.ugr.es \\ rbnuria6@gmail.com}

%-----------------------------------------------------------------------------------------------------
%	RESUMEN
%-------------------------------					----------------------------------------------------------------------

% Resumen del documento. Va en la portada.
% Puedes también dejarlo vacío, en cuyo caso no aparece en la portada.
%\newcommand{\docabstract}{}
\newcommand{\docabstract}{}

\begin{document}

 \maketitle

%-----------------------------------------------------------------------------------------------------
%	ÍNDICE
%-----------------------------------------------------------------------------------------------------

% Profundidad del Índice:
%\setcounter{tocdepth}{1}

\newpage
\tableofcontents
\newpage
	
%----------------------------------------------------------------------------------------
%	Sección 1: Deficiones y teoremas
%----------------------------------------------------------------------------------------

\section{Clases esférica y elíptica de distribuciones en $\mathbb{R}^p$}
	
	\begin{definition}
		Dado un vector aleatorio \textbf{X} = $(X_1, ... , X_p)^t$, se dice que se distribuye en la clase esférica de distribuciones en $\mathbb{R}^p$ si \textbf{X} y \textbf{HX} tienen la misma distribución, $\forall$\textbf{H} $\in O(p)$ siendo $O(p)$ el grupo de matrices ortogonales de orden p. Esto es, si su distribución es invariante frente a transformaciones ortogonales. 
	\end{definition}

	

	\begin{example}
		Un ejemplo de densidad esférica es la distribución uniforme en la hiperesfera de radio r:
		
		\begin{center}
		\begin{math}
				f_x(\textbf{x}) = \frac{\Gamma(\frac{p+2}{2})}{\pi ^ {p/2} r^p} \textbf{I}_{[\textbf{x}^t\textbf{x} \leq r^2]}
		\end{math}
		\end{center}
	\end{example}


	Vemos ahora un resultado sobre la forma que adopta la función característica de cualquier variable que se distribuya en la clase esférica.
	
	\begin{theorem}
		Sea \textbf{X} un vector aleatorio con distribución en la clase esférica. Entonces su función característica es de la forma $\phi($\textbf{t}$) = \psi$(\textbf{t}$^t$\textbf{t}), con $\psi$ una cierta función. Además, E\textbf{[X] = 0} y Cov[\textbf{X}]  = $-2\psi'(0)$\textbf{I}$_p$.
	\end{theorem}

	\begin{proof}
		EJERCICIO.
		
		Para esta demostración es necesario utilizar algunos aspectos de la teoría de invarianza (buscarlos en el apéndice A y ponerlos).
	\end{proof}

	\begin{definition}
		Dado un vector aleatorio \textbf{X} = $(X_1, ..., X_p)^t$, se dice que se distribuye en la clase elíptica de parámetros $\mu \in \mathbb{R}^p$ y \textbf{V}$_{p \times p}$ (\textbf{V} $> 0$) si su densidad es de la forma
		
		\begin{center}
		\begin{math}
			f(\textbf{x}) = C_p|\textbf{V}|^{-\frac{1}{2}}h((\textbf{x}-\mu)^t\textbf{V}^{-1}(\textbf{x}-\mu))
		\end{math}
		\end{center}
	\end{definition}

	\begin{example}
		Un ejemplo de densidad elíptica es la distribución uniforme en el elipsoide de dimensión p:
		
		\begin{center}
			\begin{math}
				f_x(\textbf{x}) = \frac{\Gamma(\frac{p+2}{2}|\textbf{A}|^\frac{1}{2})}{\pi^\frac{p}{2}r^0}\textbf{I}_{[(\textbf{x}-\mu)^t\textbf{A}(\textbf{x}-\mu)\leq r^2]}
			\end{math}
		\end{center}

	\end{example}

	\texttt{Ejercicios}
		\begin{itemize}
			\item Comprobar que la clase esférica coincide con la elíptica $E(0; I_p )$.
			
			\texttt{Sol:} Si tenemos que un vector aleatorio \textbf{X} = $(X_1, ..., X_p)^t$ se distribuye en la clase elíptica $E(0; I_p )$, entonces cumple que su función de densidad es de la forma $f_x(\textbf{x}) = C_p h (x^tx)$, por lo que sólo depende de x a través de $x^tx$. De este modo, es invariantes por transformaciones ortogonales con lo que pertenece también a la clase esférica. ¿¿¿¿El OTRO CASO???
			
			\item Verificar la siguiente caracterización, que generaliza la vista en el caso normal: $$ X \in E_p(\mu; V) \leftrightarrows X = \mu + CU $$ donde $\mu \in \mathbb{R}^p$ , C es la matriz dada por la descomposición de Cholesky de V, es decir, $V=CC^t$ y $U$ es un vector de la clase esférica p-dimensional.
			
			\item Verificar que si $X \in E_p(\mu; V)$  entonces $E[X] = \mu + CE[U]$ y $Cov(X)= C Cov(U)C^t$ siendo U pertenciente a la clase esférica.
			
			\texttt{Sol:}En este caso, aplicando el ejercicio anterior tenemos que  $X=\mu + CU$, y, por tanto, por las propiedades de linealidad de la esperanza y de la covarianza, se cumplen las dos condiciones que se piden en este apartado. ???
		\end{itemize}
	MUCHOS MAS EJERCICIOS
	
	\section{Algunas distribuciones de las clases esférica y Elíptica de distribuciones en $\mathbb{R}^p$ }
	
	\subsection{Distribución uniforme en el círculo de radio r}
	
	Sea $S=\{(x_1, x_2) \in \mathbb{R}^2: x_1^2+x_2^2 \leq r^2\}$ el círculo centrado en el origen y de radio $r>0$. Dado un vector aleatorio: \textbf{X} = $(X_1, X_2)^t$ se dice que sigue la distribución uniforme en $S^1$ si su densidad es $$f(x_1,x_2) = KI_{S^1}$$ donde $K>0$ e $I_{S^1}$ es la función indicadora en $S^1$.
	
	\texttt{Ejercicios}
	
	\begin{itemize}
		\item Verificar que $K=\frac{1}{\pi r^2}$
		
		\texttt{Sol:} Tengamos en cuenta que $\int_{-\infty}^{\infty} f_X(X) dX = 1$. Por tanto $K \int_{-\infty}^{\infty} I_{S^1} dX =1 $. Así: $$K \int_{S^1}^{}1dX = 1 $$ Y por tanto, atendiendo a que esa integral coincide con el área de un círculo: $K\pi r^2 = 1$. Y como conclusión sacamos que $K=\frac{1}{\pi r^2}$
		
		\item Comprobar que $E[X]=0$ y que $Cov[X] = \frac{r^2}{4}I_2$ ???????? I2???*********
		
		\item Calcular las distribuciones marginales así como las condicionadas. Por simetría basta calcular la distribución de $X_1$  y la de $X_1|X_2=x2$. 
		
		 
	\end{itemize} 
	
	
	\subsection{Distribución uniforme en la esfera de radio r}
	
	Sea $S^2=\{(x_1, x_2, x_3) \in  \mathbb{R}^3: x_1^2 + x_2^2+x_3^2\leq r^2\}$ el interior y borde de la esfera centrada en 0 y de radio $r>0$. Dado un vector aleatorio \textbf{X} = $(X_1, X_2, X_3)^t$ se dice que sigue la distribución uniforme en $S^2$ si su densidad es:
	$$ f(x_1, x_2, x_3)= KI_{S^2}$$
	
	donde $K>0$ e $I_{S^2}$ en la función indicadora en $S^2$. 
	
	\textbf{Ejercicios}
	\begin{itemize}
		\item Verificar que $K=\frac{3}{4\pi r^3}$
		
		\texttt{Sol:} De manera similar al caso del círculo, ahora tenemos que se cumple que $\int_{-\infty}^{\infty} f_X(X) dX = 1$. Por tanto $K \int_{-\infty}^{\infty} I_{S^2} dX =1 $. Así: $$K \int_{S^2}^{}1dX = 1 $$ Y por tanto, atendiendo a que esa integral coincide con el área de una esfera: $K\frac{4}{3}\pi r^3 = 1$. Y como conclusión sacamos que $K=\frac{3}{4\pi r^3}$
		
		\item   Comprobar que $E[X]=0$ y que $Cov[X] = \frac{r^2}{5}I_3$ ???????? 3???*********
		
		\item Calcular las distribuciones marginales unidimensionales y bidimensionales. Dada la simetría, basta con calcular la distribución de $X_1$ y de $(X_1, X_2)^t$
		
		\item Calcular las distribuciones de $(X_2, X_3)^t|X_1=x_1$ y $X_3|X_1=x_1, X_2=x_2$. Deducir que son distribuciones uniformes en un círculo y en un intervalo, respectivamente y, en consecuencia, calcular los dos primeros momentos.
		
	\end{itemize}
	
	\section{Distribución uniforme en la hiperesfera de radio r}
	
	Sea $S^{p-1} = \{x \in \mathbb{R}^p: x^tx\leq r^2\}$ el interior y el borde de la hiperesfera centrada en el origen y de radio $r>0$. Dado un vector aleatorio \textbf{X} = $(X_1, \dots , X_p)^t$ se dice que sigue la distribución uniforme en $S^{p-1}$ si su densidad es:
	
	$$f(x_1, \dots, x_p)= KI_{S^{p-1}}$$
	
	donde $K>0$ e $I_{S^{p-1}}$ es la función indicadora en $ S^{p-1}$
	
	************EJERCICIOS**************** (me da miedo hasta copiarlos xdxdxdxdxdxdxdxd)
	
	\section{Distribución T-Student esférica}
	
	Sea \textbf{X} = $(X_1, \dots , X_p)^t$ un vector aleatorio. Diremos que se distribuye según la distribución t de student esférica p-dimensional con n grados de libertad si su función de densidad es:
	$$ f(x_1, \dots, x_p)^t = \frac{\Gamma(\frac{n+p}{2})}{\Gamma(\frac{n}{2})(n \pi )^{\frac{p}{2}}[1+\frac{1}{n}x^tx]^{\frac{n+p}{2}}} $$ con $x \in \mathbb{R}^p$
	
	\textbf{Ejercicio:}Demostrar que los momentos de esta distribución son:
	
	$$ E[X]=0$$ $$ Cov[X]= \frac{n}{n-2}I_{p}$$ con $n>2$. 
	
	\section{Versiones elípticas de las densidades uniformes en la hiperesfera y t-student}
	
	**********TODO LO QUE QUEDAN SON EJERCICIOS************
	
\end{document}